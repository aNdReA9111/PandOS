\documentclass{article}

\title{\textmu PandOS}
\date{}
\usepackage{geometry}
\usepackage{hyperref}
\usepackage{xcolor}
\usepackage{titlesec}

\setcounter{secnumdepth}{4} % Imposta la profondità della numerazione delle sezioni
\setcounter{tocdepth}{4} % Set the depth for table of contents

% Definizione di un nuovo comando per \subsubsubsection
\titleclass{\subsubsubsection}{straight}[
\subsection{]}
\newcounter{subsubsubsection}
\renewcommand{\thesubsubsubsection}{\thesubsubsection.\arabic{subsubsubsection}}
\titleformat{\subsubsubsection}
{\normalfont\normalsize\bfseries}{\thesubsubsubsection}{1em}{}
\titlespacing*{\subsubsubsection}
{0pt}{3.25ex plus 1ex minus .2ex}{1.5ex plus .2ex}

\makeatletter
\def\toclevel@subsubsubsection{4}
\def\l@subsubsubsection{\@dottedtocline{4}{7em}{4em}}
\makeatother

% Impostazione dei margini
\geometry{left=3cm, right=3cm, top=2.5cm, bottom=2.5cm,}

\begin{document}
	\maketitle

	\tableofcontents
	\newpage
	\section{Cos'è?}
	µPandOS è un sistema operativo microkernel sviluppato per fini didattici; in particolare
	questa implementazione è fatta al fine di svolgere un progetto propedeutico all'esame
	per il corso \texttt{08574 - Sistemi Operativi} (anno accademico 2023/24) per
	l'università di Bologna.\\

	\section{Fase 1 - Definizione operazioni su liste di pcb e messaggi}
	\subsection{Obiettivi}
	In questa fase andremo a scrivere le basi per quanto riguarda questo progetto,
	ovvero definiremo i metodi di due strutture fondamentali per quanto riguarda
	PandOS, ossia i messaggi e i PCB

	\subsection{Prototipi delle funzioni}
	\subsubsection{Allocazione e deallocazione dei PCB:}
	\begin{itemize}
		\item \texttt{void initPcbs():} tramite la funzione \texttt{freePcb},
			vengono aggiunti in coda gli elementi della \texttt{pcbTable} (da 1 a
			MAXPROC) nella lista dei processi liberi;

		\item \texttt{void freePcb(pcb\_t *p):} mette l'elemento puntato da \texttt{p}
			nella lista dei processi liberi;

		\item \texttt{pcb\_t *allocPcb():} rimuove il primo elemento dei processi
			liberi, inizializza tutti i campi e ritorna un puntatore ad esso.
	\end{itemize}

	\subsubsection{PCB Queue:}
	\begin{itemize}
		\item \texttt{void mkEmptyProcQ(struct list\_head *head):} inizializza una
			variabile come puntatore alla testa della coda dei processi;

		\item \texttt{int emptyProcQ(struct list\_head *head):} se la coda la cui
			testa è puntata da \texttt{head} è vuota ritorna TRUE, altrimenti FALSE;

		\item \texttt{void insertProcQ(struct list\_head *head, pcb\_t *p):}
			inserisce il PCB puntato da \texttt{p} in fondo alla coda dei processi puntata
			da \texttt{*head};

		\item \texttt{pcb\_t *headProcQ(struct list\_head *head):} ritorna NULL se
			la coda dei processi è vuota, altrimenti il PCB in testa;

		\item \texttt{pcb\_t *removeProcQ(struct list\_head *head):} rimuove la
			testa della coda dei processi puntata da \texttt{*head} e ritorna un puntatore
			dell'elemento in questione; se la lista è vuota ritorna NULL;

		\item \texttt{pcb\_t *outProcQ(struct list\_head *head, pcb\_t *p):} cerca
			mediante un \texttt{for\_each} il PCB \texttt{p} nella lista puntata da
			\texttt{head} e lo rimuove; se lo trova ritorna \texttt{p} stesso,
			altrimenti NULL.
	\end{itemize}
	\newpage
	\subsubsection{PCB Trees:}
	\begin{itemize}
		\item \texttt{int emptyChild(pcb\_t *p):} ritorna l'esito della chiamata
			alla funzione \texttt{list\_empty}, alla quale viene passato come
			parametro l'indirizzo del \texttt{list\_head} \texttt{p\_child} di \texttt{p};

		\item \texttt{void insertChild(pcb\_t *prnt, pcb\_t *p):} si assegna \texttt{prnt}
			al puntatore \texttt{p\_parent} di \texttt{p}. Dopo si aggiunge \texttt{p}
			alla lista dei fratelli, tramite \texttt{list\_add} (se non ci sono altri
			figli) e \texttt{list\_add\_tail} (per rispettare la FIFOness), alle quali
			viene passato come parametro gli indirizzi del \texttt{list\_head} \texttt{p\_sib}
			di \texttt{p} e del \texttt{list\_head} \texttt{p\_child} di \texttt{prnt}
			(\texttt{p} diventa fratello dei figli di \texttt{prnt} e quindi figlio di
			\texttt{prnt}).

		\item \texttt{pcb\_t *removeChild(pcb\_t *p):} il controllo sulla presenza o
			meno di figli avviene con la funzione \texttt{emptyChild}. Se ci sono
			figli, si sceglie il primo figlio tramite la macro \texttt{container\_of},
			chiamata sull'elemento successivo al \texttt{list\_head} \texttt{p\_child}.
			In seguito il figlio viene eliminato tramite la funzione \texttt{list\_del}
			e viene troncato il legame con il padre, assegnando il valore NULL al
			puntatore \texttt{p\_parent} del figlio.

		\item \texttt{pcb\_t *outChild(pcb\_t *p):} se \texttt{p} ha un padre,
			rimuovo \texttt{p} dalla lista dei suoi fratelli chiamando \texttt{list\_del}
			a cui passo come parametro l'indirizzo di \texttt{p\_sib} di \texttt{p}, in
			seguito rimuovo il legame con il padre assegnando NULL al puntatore \texttt{p\_parent}
			di \texttt{p}.
	\end{itemize}

	\subsubsection{Allocazione e deallocazione dei messaggi:}
	\begin{itemize}
		\item \texttt{void freeMsg(msg\_t *m):} Inserisce l’elemento puntato da
			\texttt{m} in testa alla lista dei messaggi.

		\item \texttt{msg\_t *allocMsg():} Ritorna \texttt{NULL} se la lista dei messaggi
			è vuota. Altrimenti rimuove un elemento dalla testa, imposta a 0 la variabile
			\texttt{m\_payload} di ogni messaggio presente nell’array \texttt{msgTable}
			e ritorna un puntatore all’elemento rimosso.

		\item \texttt{void initMsgs():} Inserisce gli elementi presenti nell’array
			\texttt{msgTable} in coda alla lista dei messaggi.
	\end{itemize}

	\subsubsection{Message}
	\begin{itemize}
		\item \texttt{void mkEmptyMessageQ(struct list\_head *head):} Inizializza
			una una lista di messaggi vuota.

		\item \texttt{int emptyMessageQ(struct list\_head *head):} Ritorna 1 se la
			lista puntata da head è vuota, altrimenti 0.

		\item \texttt{void insertMessage(struct list\_head *head, msg\_t *m):}
			Inserisce il messaggio puntato da \texttt{m} in coda alla lista puntata da
			\texttt{head}.

		\item \texttt{void pushMessage(struct list\_head *head, msg\_t *m):}
			Inserisce il messaggio puntato da \texttt{m} in testa alla lista puntata da
			\texttt{head}.

		\item \texttt{msg\_t *popMessage(struct list\_head *head, pcb\_t *p\_ptr):}
			Rimuove il primo messaggio trovato nella lista puntata da \texttt{head} che
			è stato inviato dal thread p\_ptr.\\ Se \texttt{p\_ptr} è NULL, ritorna il
			primo messaggio in coda. \\ Se \texttt{head} è vuota o se non viene trovato
			alcun elemento mandato dal thread \texttt{p\_ptr}, ritorna null.

		\item \texttt{msg\_t *headMessage(struct list\_head *head):} Se la lista
			puntata da \texttt{head} è vuota ritorna NULL, altrimenti ritorna il
			messaggio in testa ad essa.
	\end{itemize}

	\newpage

	\section{Fase 2 - Definizione del Nucleo, Scheduler, SSI, Interrupt ed eccezioni}
	Di seguito sono riportate le scelte progettuali per quanto riguarda i moduli sviluppati:

	\subsection{Utility}
	\subsubsection{\texttt{\textbf{timer.c}}}
	In questo modulo abbiamo delle funzioni/procedure ausiliarie richiamate degli altri
	moduli per la gestione dei vari timer:
	\begin{itemize}
		\item \texttt{\textbf{unsigned int getTOD()}}: ritorna il valore del time of
			day clock, che viene nel nostro caso salvato nella variabile globale
			\texttt{\textbf{start}}: utilizzata per il calcolo del CPU time.

		\item \texttt{\textbf{void updateCPUtime(pcb\_t *p)}}: chiama la funzione qui
			sopra descritta per aggiornare il valore del campo \texttt{\textbf{p\_time}}
			del processo passato alla funzione.

		\item \texttt{\textbf{void setIntervalTimer(unsigned int t)}}: funzione che imposta
			il valore dell'interval timer.

		\item \texttt{\textbf{void setPLT(unsigned int t)}}: funzione che imposta il
			valore del processor local timer.

		\item \texttt{\textbf{unsigned int getPLT()}}: funzione che permette di ottenere
			il valore del processor local timer.
	\end{itemize}

	\subsection{Inizializzazione nucleo}
	\subsubsection{Dichiarazione e inizializzazione variabili globali}
	Nel modulo \texttt{\textbf{initial.c}} viene implementato il \textbf{main()},
	la dichiarazione delle variabili globali:
	\begin{itemize}
		\item \texttt{\textbf{int process\_count}} ossia il contatore dei processi
			attivi;

		\item \texttt{\textbf{int soft\_blocked\_count}} ossia il contatore dei
			processi bloccati;

		\item \texttt{\textbf{int start}}, variabile globale utilizzata per calcolare
			il CPU time di ogni processo che viene eseguito.

		\item \texttt{\textbf{int pid\_counter}}, usato per assegnare in maniera sequenziale
			i PID ai processi man mano che vengono creati;

		\item \texttt{\textbf{pcb\_t *current\_process}} ossia il puntatore al PCB
			del processo corrente;

		\item \texttt{\textbf{pcb\_t *ssi\_pcb}}, che è il puntatore al PCB del SSI;
	\end{itemize}

	\subsubsection{Dichiarazione e inizializzazione strutture dati}
	Vengono inoltre implementate le strutture dati principali:
	\begin{itemize}
		\item attraverso le funzioni \texttt{\textbf{initPcbs()}} e \texttt{\textbf{initMsgs()}}
			vengono inizializzate le strutture della fase 1;

		\item \texttt{\textbf{Ready\_Queue}}, ossia la lista dei proessi pronti ad essere
			eseguiti;

		\item 8 liste per i processi bloccati in attesa dei device o per il terminale
			(una per input e una per output);

		\item \texttt{\textbf{void initPassupVector()}} è una procedura che viene
			richiamata per definire il \texttt{pass up vector}, ossia è la struttura
			dati a livello hardware che indica a quale funzione passare il controllo
			quando si verifica un interrupt.
	\end{itemize}
	\subsubsection{Interval timer}
	Viene caricato l'interval timer a 100 ms attraverso la chiamata alla procedura
	ausiliaria \texttt{\textbf{setIntervalTimer(PSECOND)}} definita in \texttt{\textbf{timers.c}}

	\newpage

	\subsubsection{Processi SSI e Test}
	Infine, prima di richiamare lo \texttt{Scheduler}, attraverso la procedura
	\texttt{\textbf{void initFirstProcesses()}} vengono inseriti nella \texttt{Ready
	Queue} i processi del SSI e del test. Questi avranno lo status settato in modo
	da avere la maschera dell'interrupt abilitata, l'interval timer abilitato e
	che siano in modalità kernel. Avranno rispettivamente pid 1 e 2.

	\subsection{Scheduler}
	Lo Scheduler è il componente che gestisce la coda dei processi pronti ad
	essere eseguiti (\textbf{Ready Queue}); la procedura principale che svolge
	tutto ciò è \texttt{\textbf{void scheduler()}}; questa parte con un controllo
	iniziale sulla \texttt{\textbf{Ready Queue}} vedendo se è vuota (con \texttt{\textbf{emptyProcQ(\&Ready\_Queue)}}):
	\begin{itemize}
		\item se non è vuota prendo il processo che deve essere preso in carico dalla
			CPU (\texttt{\textbf{current\_process}}) con la funzione \texttt{\textbf{removeProcQ(\&Ready\_Queue)}},
			setto il Timer attraverso la funzione \texttt{\textbf{setPLT()}} a 5 ms (con
			la costante \textbf{TIMESLICE}) per implementare il Round Robin, e infine viene
			caricato lo stato del processo corrente nel processore (con \texttt{\textbf{LDST()}});

		\item altrimenti (se vuota), si effettua la Deadlock detection; in particolare
			può decidere se effettuare un \texttt{\textbf{HALT()}} quando non ci sono
			più processi da eseguire; se ci sono altri PCB entrerà in \texttt{\textbf{WAIT()}};
			se la ready queue è vuota e ci sono processi bloccati si entra in deadlock
			invocando \texttt{\textbf{PANIC()}} fermando così l'esecuzione;
	\end{itemize}

	\subsection{SSI}
	Essendo che \textmu PandOS è un microkernel, le uniche syscall implementate sono
	la Send e la Receive; queste vengono usate dai processi per chiedere al
	processo SSI risorse; quanto detto è implementato nell'apposito modulo \texttt{\textbf{ssi.c}},
	in particolare nella funzione \texttt{\textbf{SSILoop()}}, che implementa il polling
	del processo SSI: questa è eternamente in attesa di ricevere un messaggio da
	un qualsiasi processo che necessità una risorsa, prova a soddisfarlo attraverso
	l'apposita funzione \texttt{\textbf{unsigned int SSIRequest(pcb\_t* sender,
	ssi\_payload\_t *payload)}} e se riesce viene inviato un riscontro al processo
	che ha effettuato la richiesta tramite la syscall send. Di seguito si forniranno
	maggiori dettagli riguardo quest'ultima funzione;

	\newpage

	\subsubsection{SSIRequest}
	All'interno di questa funzione vengono analizzati i parametri \texttt{\textbf{pcb\_t*
	sender, ssi\_payload\_t *payload}} che contengono rispettivamente il processo che
	ha richiesto il servizio e il messaggio mandato col servizio richiesto; in
	particolare nel messaggio è determinante il \texttt{\textbf{service\_code}},
	che stabilisce l'oggetto della richiesta del sender. Attraverso l'utilizzo di uno
	switch su quest'ultimo parametro, la ssi fornisce il servizio richiesto. Si distinguono
	in particolare i casi:
	\begin{itemize}
		\item 1 (\texttt{\textbf{CREATEPROCESS}}): viene richiesta la creazione di un
			processo; questa richiesta viene soddisfatta solo se c'è spazio nella
			tabella dei processi liberi; in caso affermativo viene invocata la funzione
			\texttt{\textbf{ssi\_new\_process()}} con parametri il sender, che fungerà
			da parent, e un puntatore a una variabile di tipo \texttt{\textbf{ssi\_create\_process}},
			contenete i dati necessari alla creazione del nuovo processo; Nella funzione
			\texttt{\textbf{ssi\_new\_process}} viene chiamata \texttt{\textbf{allocPcb()}}
			per l'allocazione in memoria del nuovo processo e successivamente questo viene
			inserito nelle \texttt{\textbf{ready queue}} e nella lista child del pcb
			sender.

		\item 2 (\texttt{\textbf{TERMPROCESS}}), viene chiamata la procedura \texttt{\textbf{ssi\_terminate\_process()}}
			passando come parametro il processo da terminare. In questa funzione avviene
			ricorsivamente la terminazione di tutti i processi figli del sender e dei realtivi
			fratelli.

		\item 3 (\texttt{\textbf{DOIO}}), viene chiamata la procedura \texttt{\textbf{ssi\_doio}}
			con parametri il puntatore al pcb del sender e un puntatore a una
			variabile di tipo \texttt{\textbf{ssi\_do\_io\_t}} contenente i dati necessari
			a tale richiesta. Viene successivamente chiamata la procedura \texttt{\textbf{addrToDevice()}}
			che, dato l'indirizzo del device passato con la richiesta di DOIO, ne determina
			i rispettivi numero di device e linea, per poi chiamare la funzione
			\texttt{\textbf{blockProcessOnDevice}}. Quest'ultima, passati come argomenti
			il pcb del sender, il numero del device e un intero che indica se si sta
			facendo una lettura o una scrittura, rimuove il sender dalla \texttt{\textbf{ready
			queue}} e lo inserisce nella lista di processi bloccati del relativo
			device. La funzione \texttt{\textbf{addrToDevice()}} invidua la corretta
			linea e il corretto numero di device confrontando il command\_address
			passato come argomento con tutti i campi di DEV\_REG\_ADDR[][]; per una maggiore
			efficienza questo viene prima confrontato con i campi relativi al terminale
			e successivamente, attraverso un ciclo for annidato, con i campi degli altri
			device.

		\item 4 (\texttt{\textbf{GETTIME}}), viene restituito come \texttt{\textbf{unsigned
			int}} un puntatore alla variabile \texttt{\textbf{p\_time}} del processo sender.

		\item 5 (\texttt{\textbf{CLOCKWAIT}}), viene chiamata la funzione \texttt{\textbf{ssi\_clockwait}}
			con argomento il puntatore alla struct \texttt{\textbf{pcb\_t}} del processo
			sender. Quest'ultima inserisce il sender nella lista dei processi bloccati
			da pseudo clock. La procedura \texttt{\textbf{pseudoClockInterruptHandler()}}
			definita in \texttt{\textbf{interrupts.c}} si occupa della gestione di questa
			lista.

		\item 6 (\texttt{\textbf{GETSUPPORTPTR}}), viene restituito come \texttt{\textbf{unsigned
			int}} un puntatore alla variabile \texttt{\textbf{p\_supportstruct}} del processo
			sender.

		\item 7 (\texttt{\textbf{GETPROCESSID}}), viene invocata la funzione \texttt{\textbf{int
			ssi\_getprocessid}}. Quest'ultima prende come parametri il puntatore al pcb
			del sender e un puntatore generico arg, e ritorna il pid del sender se l'argomento
			è NULL, altrimenti il pid del processo padre del chiamante.

		\item Se il service code non contiene nessuno dei seguenti codici viene terminato
			il sender con la funzione \texttt{\textbf{ssi\_terminate\_process()}}.
	\end{itemize}

	In alcune delle funzioni menzionate vengono inoltre modificati i valori di variabili
	globali quali:
	\begin{itemize}
		\item \texttt{\textbf{soft\_blocked\_count}} : viene decrementata in \texttt{\textbf{ssi\_terminate\_process()}}
			se il processo viene rimosso da una delle liste di processi bloccati dei device,
			o incrementata in \texttt{\textbf{ssi\_clockwait()}} quando viene aggiunto.

		\item \texttt{\textbf{process\_count}} : viene incrementata alla creazione
			di un nuovo processo in \texttt{\textbf{ssi\_create\_process}} e decrementata
			quando un processo viene terminato in \texttt{\textbf{ssi\_terminate\_process()}}.
	\end{itemize}

	\newpage
	\subsection{Gestore delle eccezioni}
	La funzione che si occupa della gestione delle eccezioni è la funzione \texttt{\textbf{void
	exceptionHandler()}} dichiarata nel file phase2/include/exceptions.h e
	definita nel file phase2/exceptions.c. Questa funzione salva lo stato al tempo
	dell'eccezione dalla \texttt{\textbf{BIOSDATAPAGE}} ed in seguito trova il
	codice dell'eccezione eseguendo operazioni di manipolazione dei bit sul
	registro cause, ottenuto con la funzione \texttt{\textbf{getCAUSE}}. In particolare
	si esegue l'operazione \texttt{\textbf{cause \& GETEXECCODE}} che permette di
	mantenere solo i bit che definiscono il codice dell'eccezione, i quali vengono
	shiftati a destra di 2 posizioni (costante \texttt{\textbf{CAUSESHIFT}}).
	\subsubsection{Interrupt Handler}
	Nel caso il codice dell'eccezione abbia valore 0 (costante \texttt{\textbf{IOINTERRUPTS}})
	viene invocata la funzione per la gestione degli interrupt \texttt{\textbf{void
	interruptHandler(int cause, state\_t* exception\_state)}}. Qui viene
	utilizzata la macro \texttt{\textbf{CAUSE\_IP\_GET(cause, line)}}, grazie alla
	quale, passando il cause register e il valore di una linea di interrupt, è
	possibile sapere se c'è un interrupt su quella linea. Il controllo viene fatto
	per tutte le linee, seguendo l'ordine di priorità che va dall'interrupt
	causato dal processor local timer, all'interrupt causato da un dispositivo terminale.
	In base alla linea su cui avviene l'interrupt viene invocato un'opportuna
	funzione per la gestione di quello specifico interrupt. \subsubsubsection{Gestione Interrupt Processor Local Timer}
	L'interrupt causato dal processor local timer si verifica quando il tempo
	nella CPU per il processo corrente si esaurisce. Per un'opportuna gestione di questo
	interrupt usiamo la funzione \texttt{\textbf{static void
	localTimerInterruptHandler(state\_t *exception\_state)}}. In questa routine viene
	riconosciuto l'interrupt con la chiamata \texttt{\textbf{setPLT(-1)}}, in seguito
	si aggiorna il CPU time del processo corrente, si copia lo stato dell'eccezione
	nello stato del processo corrente, il quale infine viene inserito sulla ready queue.
	Dopo queste operazioni viene chiamato lo scheduler. \subsubsubsection{Gestione Interrupt Interval Timer}
	In questo caso l'ACK dell'interrupt è eseguito con la chiamata \texttt{\textbf{setIntervalTimer(PSECOND)}}.
	Dopodiché avvienelo sblocco di tutti i processi che erano in attesa dell'interrupt,
	rimuovendo ciascuno di essi dalla lista \texttt{\textbf{Locked\_pseudo\_clock}},
	inserendoli sulla ready queue, dopo aver inviato loro un messaggio che
	consentirà ai processi interessati di sbloccarsi, quando rieseguiranno la SYS2
	su cui si erano precedentemente bloccati. Ogni volta che viene rimosso un
	processo dalla lista dei processi in attesa dello pseudoclock tick, viene decrementata
	la variabile globale \texttt{\textbf{soft\_blocked\_count}}. Infine, se il
	processo corrente è diverso da NULL, si esegue una LDST con lo stato dell'eccezione,
	altrimente viene chiamato lo scheduler. \subsubsubsection{Gestione Interrupt Device}
	La gestione degli interrupt legati a tutti gli altri device viene affidata
	alla funzione \texttt{\textbf{static void deviceInterruptHandler(int line, int
	cause, state\_t *exception\_state)}}, la quale ricava la bitmap degli
	interrupt per i dispositivi della linea d'interesse. Questo viene realizzato accedendo
	all'area di memoria riservata ai device, all'inidirizzo \texttt{\textbf{BUS\_REG\_RAM\_BASE}}.
	In seguito si esegue l'and sui bit della bitmap con le costanti \texttt{\textbf{DEVXON}}
	con X $\in \{0, \ldots, 7\}$, con questa operazione si ottiene il numero del
	device sulla line che ha causato l'interrupt, per l'ordine con cui queste operazioni
	sono effettuate,il numero del device calcolato sarà sempre quello a priorità maggiore.
	Calcolato il numero, data la linea si sblocca il processo dalla lista associata
	alla linea cercandolo tramite il device number, grazie al campo aggiuntivo
	\texttt{\textbf{dev\_no}} che abbiamo messo ai pcb. Questo campo viene settato
	dall'ssi quando viene bloccato il processo in attesa di interrupt durante il
	servizio DOIO. In caso di interrupt causato da un dispositivo terminale verifichiamo
	che il codice dell'operazione di trasmissione sia uguale a 5 (interrupt in
	attesa di essere riconosciuto), se così è allora significa che l'operazione è
	un'operazione di trasmissione di un carattere, altrimenti si tratta di un'operazione
	di ricezione. Per accedere al registro del device usiamo la macro \texttt{\textbf{DEV\_REG\_ADDR}}
	nel modo seguente: \\ \texttt{\textbf{dtpreg\_t *device\_register = (dtpreg\_t
	*)DEV\_REG\_ADDR(line, device\_number);}} \\ In caso di dispositivi terminali,
	l'operazione è analoga: \\ \texttt{\textbf{termreg\_t *device\_register = (termreg\_t
	*)DEV\_REG\_ADDR(line, device\_number);}} \\ L'operazione di riconoscimento
	dell'interrupt avviene con l'istruzione\\ \texttt{\textbf{device\_register->command
	= ACK;}} \\per i terminali a seconda di quale subdevice ha generato l'interrupt:
	\\ \texttt{\textbf{device\_register->transm\_command = ACK;}} \\ oppure \\ \texttt{\textbf{device\_register->recv\_command
	= ACK;}}.

	L'accesso allo status avviene con l'operazione seguente: \texttt{\textbf{device\_register->status}}.
	\\ Per i terminali: \\ \texttt{\textbf{device\_register->recv\_status}}. \\
	\texttt{\textbf{device\_register->transm\_status}}.

	Infine se il processo sbloccato è diverso da NULL, si mette lo status nel suo
	registro v0, gli viene inviato un messaggio avente la ssi come mittente e come
	payload lo status del device, si inserisce il processo sulla ready queue e si
	diminuisce di un'unità \texttt{\textbf{soft\_blocked\_count}}. In seguito, se il
	processso corrente è diverso da NULL, si chiama lo scheduler, altrimenti si esegue
	una LDST dello stato ottenuto dalla \texttt{\textbf{BIOSDATAPAGE}}.
	\subsubsection{Pass Up or Die}
	Tramite Pass Up or Die il kernel gestisce tutte le eccezione che non sono syscall
	o interrupt, abbiamo quindi implementato un'apposita funzione \texttt{\textbf{static
	void passUpOrDie(int i, state\_t *exception\_state)}} che controlla che la support
	struct del processo corrente sia diversa da NULL, se ciò è vero allora si
	salva lo stato dell'eccezione nello stato corretto della struttura di supporto:
	\\ \texttt{\textbf{saveState(\&(current\_process->p\_supportStruct->sup\_exceptState[i]),
	exception\_state);}} \\ ed in seguito si esegue LDCTX, passando come parametri
	i valori del giusto constesto della struttura di supporto: \\ \texttt{\textbf{
	LDCXT(current\_process->p\_supportStruct->sup\_exceptContext[i].stackPtr, \\ current\_process->p\_supportStruct->sup\_exceptContext[i].status,
	\\ current\_process->p\_supportStruct->sup\_exceptContext[i].pc \\ ); }} \\ L'indice
	i, parametro della funzione, può assumere due valori a seconda del tipo di eccezione:
	\begin{itemize}
		\item \texttt{\textbf{GENERALEXCEPT}}: per trap generiche, con codici 4...7
			e 9...12;

		\item \texttt{\textbf{PGFAULTEXCEPT}}: per eccezioni TLB, con codici 1...3;

			In caso di puntatore nullo, chiamiamo la funzione per la terminazione dei processi,
			che viene utilizzata dall'ssi per fornire il servizio \texttt{\textbf{TERMINATEPROCESS}}.
	\end{itemize}
	\newpage
	\subsubsection{Eccezioni causate da SYSCALL}
	L'eccezioni da SYSCALL sono quelle il cui codice ha valore \texttt{\textbf{SYSEXCEPTION}}.
	Quando queste si verificano, viene chiamata la funzione \texttt{\textbf{static
	void syscallExceptionHandler(state\_t *exception\_state)}}. \\In primis la
	funzione controlla il valore del bit KUc, se questo è 1 (user-mode) viene generata
	una trap generica, in cui il valore del codice dell'eccezione viene settato a
	\texttt{\textbf{EXC\_RI}} (istruzione riservata). Se il processo è in kernel-mode
	allora si continua e si controlla il registro a0 per capire quale servizio è
	stato richiesto:
	\begin{itemize}
		\item \texttt{\textbf{reg\_a0 = SENDMESSAGE}}: il processo corrente vuole eseguire
			una \texttt{\textbf{SYS1}}, dal registro a1 viene estratto l'indirizzo del
			PCB al quale si vuole inviare un messaggio, si controlla se il
			destinatario è nella ready queue o se è il processo corrente, in entrambi questi
			casi si inserisce il messaggio nell'inbox del destinatario. Se il
			destinatario è sulla lista dei processi liberi, allora esso è inesistente e
			si aggiorna solamente il registro v0 al valore \texttt{\textbf{DEST\_NOT\_EXISTS}}.
			Se il processo destinatario non si trova sulla ready e queue e nemmeno nella
			lista dei processi liberi, allora dopo l'invio del messaggio, lo si inserisce
			nella ready queue (se il messaggio inviato non è quello per cui era in
			attesa, il processo si bloccherà di nuovo). La funzione che invia i messaggi
			è \texttt{\textbf{int send(pcb\_t *sender, pcb\_t *dest, unsigned int
			payload)}}. In questo caso come payload viene passato il valore del registro
			a2. Infine si incrementa il \texttt{\textbf{PC}} e si esegue \texttt{\textbf{LDST(exception\_state)}}.

		\item \texttt{\textbf{reg\_a0 = RECEIVEMESSAGE}}: il processo corrente vuole
			eseguire una \texttt{\textbf{SYS2}}, dal registro a1 si accede all'indirizzo
			del PCB dal quale il processo richiedente vuole ricevere un messaggio,
			questo valore viene passato, insieme alla inbox del processo corrente,
			alla funzione \texttt{\textbf{popMessage}}, la quale restituirà il
			messaggio o \texttt{\textbf{NULL}} se non lo trova. Nel primo caso la SYS2
			non è bloccante, quindi si assegna al registro v0 l'indirizzo di chi ha inviato
			il messaggio, e si mette nell'area di memoria puntata dal registro a2 il
			payload del messaggio. Si incrementa il \texttt{\textbf{PC}} e si esegue
			\texttt{\textbf{LDST(exception\_state)}}, poiché non è una syscall
			bloccante. Se invece la \texttt{\textbf{popMessage}} ha restituito \texttt{\textbf{NULL}},
			allora la syscall sarà bloccante, lo stato dell'eccezione viene salvato nello
			stato del processo corrente, si aggiorna il CPU time del processo corrente
			ed infine si chiama lo scheduler.

		\item \texttt{\textbf{reg\_a0 >= 1}}: viene generata una trap generica.
	\end{itemize}

	\newpage

	\section{Fase 3 - Il livello di supporto}
	\subsection{Inizializzazione}
	L'inizializzazione viene realizzata nel file initProc.c, in questo file il
	primo processo si occupa di creare tutti i processi necessari e di
	inizializzare le strutture dati necessarie al livello di supporto. Il codice di
	inizializzazione si trova nel file phase3/initProc.c, l'header corrispondente
	si trove nel file phase3/include/initProc.h .
	\subsubsection{Inizializzazione Swap Pool Table}
	La funzione che si occupa di questa operazione impostra l'asid di ogni entry
	della tabella al valore costante \texttt{\textbf{NOPROC}}.
	\subsubsection{Inizializzazione U-proc}
	In seguito viene realizzata l'inizializzazione delle strutture di supporto per
	gli U-proc. Per farlo allochiamo consecutivamente blocchi di memoria di dimensione
	fissa \texttt{\textbf{PAGESIZE}}. La memoria viene allocata a partire dall'indirizzo
	\texttt{\textbf{RAMTOP - 3 * PAGESIZE}}, ovvero il primo indirizzo disponibile,
	poiché i primi tre frame sono occupati dal processo ssi e dal processo test. In
	questa funzione, vengono assegnati alle support struct gli indirizzi dei
	gestori delle eccezioni del livello di support (\texttt{\textbf{pager}} e
	\texttt{\textbf{generalExceptionHandler}}). Nel frattempo anche gli stati dei vari
	u-proc vengono inizializzati, in particolare per questi processi si usa il
	medesimo indirizzo per lo stack pointer(\texttt{\textbf{USERSTACKTOP}}) e il
	medesimo indirizzo per il program counter (\texttt{\textbf{USERPROCSTARTADDR}}),
	in quanto essi fanno uso di indirizzo logici. Le strutture di supporto e gli stati
	vengono memorizzati in array, affinché gli sst possano poi creare gli u-proc, reperendo
	lo stato e la struttura di supporto corretti con facilità. Infine per ciascuna
	struttura di supporto avviene l'inizializzazione della rispettiva tabella
	delle pagine, impostando per ogni entry della tabella i valori per \texttt{\textbf{pte\_entryHI}}
	(calcolo indirizzo logico) e \texttt{\textbf{pte\_entryLO}} (accesione dirty
	bit).
	\subsubsection{Processo swap mutex}
	Per mantenere mutua esclusione sulla swap pool table si utilizza un processo
	che realizza tramite message passing un semaforo binario. Nella funzione \texttt{\textbf{initSwapMutex()}}
	viene allocata memoria per questo processo e al suo program counter viene
	assegnando l'indirizzo della funzione \texttt{\textbf{swapMutex}}, che
	implementa il semaforo eseguendo:
	\begin{enumerate}
		\item Receive senza mittente specificato (in questo modo il semaforo riceve
			le richieste da tutti i processi)

		\item Send al mittente del messaggio ricevuto nel punto 1. (questo messaggio
			concede la mutua esclusione)

		\item Receive dal mittente del messaggio ricevuto nel punto 1. (il semaforo
			attende il rilascio della mutua esclusione)
	\end{enumerate}
	\subsubsection{Inizializzazione processi device}
	Questi processi sono processi che ricevono dal rispettivo sst la stringa da stampare
	in output e questi realizzano un loop in cui ad ogni iterazione viene inviata una
	richiesta alla ssi per il servizio \texttt{\textbf{DOIO}} per la stampa del
	carattere della stringa in questione. Di questi processi ne sono stati realizzati
	16 (8 terminali e altrettante stampanti). Eseguono tutti lo stesso codice, consistente
	in una funzione con segnatura \texttt{\textbf{void print(int device\_number, unsigned
	int *base\_address)}} che calcola i registri del dispositivo partendo da
	\texttt{\textbf{base\_address}} (indirizzo della prima stampante o del primo
	terminale) calcola l'indirizzo del dispositivo in questione utilizzando la seguente
	formula:
	\newline
	\texttt{\textbf{base\_address + device\_number * 4}}
	\newline
	Poi dall'indirizzo calcolato si calcolano il registo di comando, e (nel caso delle
	stampanti) anche il registro data0 nel quale andare ad inserire il carattere
	della stringa da stampare. Una volta realizzate queste operazioni preliminari può
	avvenire la ricezione della stringa e la conseguente procedura di I/O.
	\newline
	In seguito per ciascun dispositivo si realizza una funzione wrapper che chiama
	la funzione parametrizzata qui sopra discussa, in modo da conferire dei nomi chiari
	e che consentano il raccoglimento di queste funzioni con un array di puntatori
	ad esse. Questo procedimento è stato implementato con il seguente codice:

	\texttt{\textbf{void print\_term0() \{ print(0, (unsigned int *)TERM0ADDR); \}}}
	\newline
	\texttt{\textbf{void print\_term1() \{ print(1, (unsigned int *)TERM0ADDR); \}}}
	\newline
	\texttt{\textbf{void print\_term2() \{ print(2, (unsigned int *)TERM0ADDR); \}}}
	\newline
	\texttt{\textbf{void print\_term3() \{ print(3, (unsigned int *)TERM0ADDR); \}}}
	\newline
	\texttt{\textbf{void print\_term4() \{ print(4, (unsigned int *)TERM0ADDR); \}}}
	\newline
	\texttt{\textbf{void print\_term5() \{ print(5, (unsigned int *)TERM0ADDR); \}}}
	\newline
	\texttt{\textbf{void print\_term6() \{ print(6, (unsigned int *)TERM0ADDR); \}}}
	\newline
	\texttt{\textbf{void print\_term7() \{ print(7, (unsigned int *)TERM0ADDR); \}}}
	\newline

	\texttt{\textbf{void printer0() \{ print(0, (unsigned int *)PRINTER0ADDR); \}}}
	\newline
	\texttt{\textbf{void printer1() \{ print(1, (unsigned int *)PRINTER0ADDR); \}}}
	\newline
	\texttt{\textbf{void printer2() \{ print(2, (unsigned int *)PRINTER0ADDR); \}}}
	\newline
	\texttt{\textbf{void printer3() \{ print(3, (unsigned int *)PRINTER0ADDR); \}}}
	\newline
	\texttt{\textbf{void printer4() \{ print(4, (unsigned int *)PRINTER0ADDR); \}}}
	\newline
	\texttt{\textbf{void printer5() \{ print(5, (unsigned int *)PRINTER0ADDR); \}}}
	\newline
	\texttt{\textbf{void printer6() \{ print(6, (unsigned int *)PRINTER0ADDR); \}}}
	\newline
	\texttt{\textbf{void printer7() \{ print(7, (unsigned int *)PRINTER0ADDR); \}}}
	\newline
	\newline
	\texttt{\textbf{ void (*terminals[8])() = \{print\_term0, print\_term1, print\_term2,
	print\_term3, print\_term4, print\_term5, print\_term6, print\_term7\}; }}
	\newline
	\texttt{\textbf{ void (*printers[8])() = \{printer0, printer1, printer2, printer3,
	printer4, printer5, printer6, printer7\};}}

	In questo modo nell'inizializzazione che seguirà sarà possibile assegnare il
	program counter accedendo agli elementi di questi array.

	\subsubsection{Inizializzazione SST}
	Dopo i processi device l'inizializzazione continua, sempre allocando la memoria
	dalla RAM, decrementando il valore di \texttt{\textbf{curr}} ad ogni iterazione.
	Anche qui si inizializza lo stato per ogni SST, in particolare il program counter
	ha valore \texttt{\textbf{SST\_loop}}, che è la funzione che implementa gli SST.

	\subsubsection{Terminazione}
	Dopo questa serie di inizializzazioni, il processo test rimane in attesa (tramite
	receive) che ciascun SST comunichi ad esso la fine della propria esecuzione. Una
	volta che tutti gli SST hanno finito l'esecuzione, il processo test richiede
	la propria terminazione all'SSI. In seguito l'SSI sarà l'unico processo nel sistema
	e si avrà lo stato di HALT.

	\newpage
	\subsection{Gestore delle eccezione a livello di supporto}
	La funzioen che si occupa della gestione delle eccezioni a livello di supporto
	è la funzione \texttt{\textbf{void generalExceptionHandler();}} che prima di
	poter realizzare una corretta gestione della trap, deve ottenere dalla SSI (servizio
	\texttt{\textbf{GETSUPPORTPTR}}) la support struct dell'u-proc interessato. Da
	questa si ottiene lo state nel modo seguente:
	\newline
	\texttt{\textbf{state\_t* eception\_state = \&(sup\_struct\_ptr->sup\_exceptState[GENERALEXCEPT]);}}.
	Dopodiché si osserva il codice dell'eccezione dal cause register in questo
	modo:
	\newline
	\texttt{\textbf{int val = (exception\_state->cause \& GETEXECCODE) >>
	CAUSESHIFT;}}
	\newline
	In base al codice ricavato viene eseguito l'handler associato. Prima di ciò viene
	incrementato il program counter dello state ottenuto in precedenza.

	\subsubsection{Gestore eccezioni causate da SYSCALL}
	In base al valore contenuto nel registro a0, uno dei due servizi sarà eseguito:
	\begin{itemize}
		\item \texttt{\textbf{reg\_a0 = SENDMSG}}: il processo corrente vuole eseguire
			una \texttt{\textbf{SYS1}}, in questo caso serve controllare il registo a1
			per verificare a chi inviare il messaggio:
			\begin{itemize}
				\item \texttt{\textbf{reg\_a1 = PARENT}}, in questo caso l'u-proc sta cercando
					di inviare un messaggio al suo SST, per farlo viene quindi eseguita l'istruzione
					seguente:

					\texttt{\textbf{SYSCALL(SENDMESSAGE, (unsigned int)current\_process->p\_parent,
					\newline
					exception\_state->reg\_a2, 0);}}

				\item \texttt{\textbf{reg\_a1 != PARENT}}, in questo caso l'u-proc sta semplicemente
					inviando un messaggio a un altro processo, per farlo viene eseguita l'istruzione
					seguente:
					\newline
					\texttt{\textbf{SYSCALL(SENDMESSAGE, exception\_state->reg\_a1,
					\newline
					exception\_state->reg\_a2, 0);}}
			\end{itemize}

		\item \texttt{\textbf{reg\_a0 = RECEIVEMSG}}: il processo corrente vuole eseguire
			una \texttt{\textbf{SYS2}}, viene quindi semplicemente eseguita questa istruzione:
			\newline
			\texttt{\textbf{SYSCALL(RECEIVEMESSAGE, exception\_state->reg\_a1, exception\_state->reg\_a2,
			0);}}
	\end{itemize}
	\subsubsection{Gestore trap generiche a livello utente}
	Qualsiasi altro codice ricavato dal cause register viene interpretato come una
	trap inaspettata, di conseguenza l'u-proc interessato verrà terminato, ma non prima
	di aver rilasciato la mutua esclusione sulla swap pool, nel caso l'avesso
	precedentemente ottenuta.

	\newpage

	\subsection{Gestione memoria virtuale}
	In questa parte, i principali attori che rendono possibile il funzionamento
	della memoria virtuale sono:
	\begin{itemize}
		\item Il TLB, una cache utilizzata dalle unità di gestione della memoria
			della CPU per velocizzare la traduzione degli indirizzi virtuali in
			indirizzi fisici;

		\item Il TLB Refill Handler: quando un indirizzo virtuale non è presente
			nella TLB si verifica quello che in gergo è definito TLB Miss, ed entra in
			gioco proprio questo componente che sposta nel TLB la pagina che ha causato
			la trap per velocizzarne accessi futuri;

		\item La swap pool table;

		\item Il processo che permette di avere mutua esclusione (swap\_mutex, discusso
			prima).
	\end{itemize}

	Il codice che implementa il meccanismo di gestione della memoria virtuale è
	nel file \texttt{\textbf{vmSupport.c}}

	\subsubsection{Pager}
	Il pager è quel componente responsabile dell'avvicendamento della pagine in
	memoria, in particolare andando a esaminare nel dettaglio l'implementazione:
	\begin{itemize}
		\item Come prima cosa, chiede la Support struct mandando un messaggio al
			processo SSI e attendendo una sua risposta;

		\item Una volta ottenuta, è necessario andare a vedere la causa dell'eccezione,
			in particolare se è una TLB-Modification exception procedo uccidendo il processo
			tramite la funzione apposita kill\_proc(), altrimenti proseguo;

		\item è fondamentale per proseguire avere mutua esclusione, poichè queste
			sono operazioni sono molto delicate e non può essere che più di un
			processo entri nel pager; la mutua esclusione si ottiene mandando un messaggio
			al processo che se ne occupa e aspettando che questo risponda concedendola;

		\item una volta ottenuta, prendo dalla support struct la pagina virtuale con
			la macro \texttt{\textbf{GET\_VPN()}}, e la pagina da rimpiazzare
			guardando prima fra quelle vuote e poi in caso facendo entrare in gioco l'algoritmo
			FIFO per selezionare una pagina vittima;

		\item dato il numero della pagina vittima è necessario calcolare il suo indirizzo,
			tenendo conto della dimensione della singola pagina e dell'offset della
			swap area:

			\begin{center}
				\texttt{\textbf{memaddr victim\_addr = (memaddr)SWAP\_POOL\_AREA + (i *
				PAGESIZE)}};
			\end{center}

		\item nel caso la pagina vittima fosse scelta dall'algoritmo di rimpiazzamento,
			è necessario aggiornarla poichè precedentemente occupata da un altro frame
			appartenente ad un altro processo; dunque serve "pulirla" poichè ormai obsoleta
			(ovviamente tutto ciò in modo atomico per evitare inconsistenza);

		\item dopo aver letto il backing store del current\_process, modifico il
			contenuto della entryHI nella swap table per far si che sia aggiornata con
			nuovi contenuti;

		\item disabilito gli interrupt momentaneamente (per non essere disturbati
			durante questo procedimento) e aggiorno la tabella delle pagine del processo
			corrente per la pagina p, indicando che la pagina è presente (V bit) e
			occupa il frame i;

		\item viene rilasciata la mutua esclusione restituendo il controllo al processo;
	\end{itemize}

	\newpage

	\subsubsection{TLB Refill Handler}
	Quando avviene un cache-miss, il TLB ha come compito quello di inserire la voce
	mancante in tabella e far ripartire dall'ultima istruzione

	\subsubsection{Algoritmo di rimpiazzamento}
	Nel caso in cui non ci siano pagine libere, (ossia scorrendo la swap table non
	troviamo nessuna pagina con asid -1 [\texttt{\textbf{NOPROC}}]), se serve
	caricarne una nuova in memoria è necessario trovare un modo per scegliere la
	pagina vittima da rimpiazzare; questa logica è implementata dalla funzione \texttt{\textbf{getPage()}},
	che dopo aver verificato che tutte le pagine siano occupate, sceglie la pagina
	da rimpiazzare tramite l'algoritmo FIFO. Di seguito si riportano le righe di codice
	che implementano tale algoritmo:

	\begin{center}
		\texttt{\textbf{static int i = -1; return i = (i + 1) \% POOLSIZE;}}
	\end{center}

	\subsubsection{RWBackingStore}
	Funzione ausiliaria che si occupa della DoIO sui vari flash device; fa ciò inserendo
	nel campo DATA0 del dispositivo flash l'indirizzo fisico iniziale del blocco da
	leggere o scrivere; successivamente utilizza il servizio DoIO del SSI per
	scrivere nel campo COMMAND del dispositivo flash; per capire se leggere o scrivere
	sulla pagina desiderata viene passato come parametro una variabile discreta (w,
	che prende come valori solo 0 e 1), in particolare è scrittura se [w = 1] o lettura
	se [w = 0].

	\subsubsection{kill\_proc}
	Funzione ausiliaria usata per gestire gli errori, che banalmente richiede la
	terminazione di un processo al SSI; prima di terminarlo libera tutte le pagine
	occupate da quest'ultimo.

	\subsubsection{cleanDirtyPage}
	Funzione ausiliaria usata dal pager che prende in input una pagina per invalidarla
	e aggiornare il TLB (in modo atomico disabilitando gli interrupt).

	\newpage
	\subsection{System Service Thread (SST)}
	Ogni System Service Thread (SST) ha lo scopo di creare il proprio processo figlio
	u-proc, e successivamente fornire a questo i servizi richiesti. Dopo la creazione
	del proprio figlio, sst rimane costantemente in attesa di richieste da parte
	di questo. Quanto detto è implementato nell'apposito modulo sst.c, in
	particolare nella funzione \texttt{\textbf{SST\_Loop()}}, che implementa il
	polling dei processi SST e chiama la funzione SSTRequest per soddisfare la
	richiesta del processo figlio.

	\subsubsection{\texttt{\textbf{SST\_loop()}}}
	In particolare \texttt{\textbf{SST\_loop()}} contiene:
	\begin{itemize}
		\item Inizializzazione delle strutture di supporto necessarie alla creazione
			del processo figlio;

		\item Chiamata alla funzione \texttt{\textbf{create\_process()}} definita in
			\texttt{\textbf{initproc.c}}, la quale crea un nuovo processo facendo una
			richiesta alla \texttt{\textbf{SSI}};

		\item While loop contente system call bloccante per l’attesa di nuove richieste,
			chiamata alla funzione \texttt{\textbf{SSTRequest}}, e system call per la
			restituzione al processo figlio di un eventuale valore di ritorno (utile per
			\texttt{\textbf{getTOD}}).
	\end{itemize}

	\subsubsection{\texttt{\textbf{SSTRequest()}}}
	Attraverso l'utilizzo di uno switch sulla variabile \texttt{\textbf{service\_code}},
	che determina il tipo di servizio richiesto da u-proc e viene passata all’interno
	della struct payload, questa funzione soddisfa la richiesta chiamando a sua
	volta apposite funzioni. Si distinguono in particolare i casi:

	\begin{itemize}
		\item \texttt{\textbf{GET\_TOD}} Viene invocata la funzione getTOD(), la quale
			consente di recuperare il numero di microsecondi trascorsi dall'ultima operazione
			del sistema avviato/ripristinato. In questo caso viene impostato come
			valore di ritorno il numero di microsecondi.

		\item {\texttt{\textbf{TERMINATE}} Viene invocata la funzione \texttt{\textbf{sst\_terminate()}}
			la quale effettua una richiesta di terminazione di u-proc alla ssi e si
			occupa di mandare un messaggio al test per comunicare la terminazione del
			processo.

		\item {\texttt{\textbf{WRITEPRINTER e WRITETERMINAL}}} Viene invocata la funzione
			\texttt{\textbf{sst\_write()}} la quale attraverso una syscall manda la
			stringa da stampare alla stampante o al terminale con lo stesso ASID del
			mittente. A \texttt{\textbf{sst\_write()}} viene passato come argomento un
			parametro chiamato \texttt{\textbf{device\_type}}, il quale determina la
			tipologia del dispositivo su cui verrà stampata la stringa passata nel
			payload (stampante o terminale). Viene quindi utilizzato uno switch su quest’ultimo
			parametro per differenziare i due casi.
	\end{itemize}

	\newpage
	\section{Crediti}
	\subsection{Github}
	Il sorgente del progetto è reperibile nella seguente
	\href{https://github.com/aNdReA9111/PandOS.git}{\textcolor{blue}{repository}}
	su Github.

	\subsection{Autori}
	\begin{itemize}
		\item Fiorellino Andrea, matricola: 0001089150,
			\href{mailto:andrea.fiorellino@studio.unibo.it}{\textcolor{blue}{andrea.fiorellino@studio.unibo.it}}

		\item Po Leonardo, matricola: 0001069156,
			\href{mailto:leonardo.po@studio.unibo.it}{\textcolor{blue}{leonardo.po@studio.unibo.it}}

		\item Silvestri Luca, matricola: 0001080369,
			\href{mailto:luca.silvestri9@studio.unibo.it}{\textcolor{blue}{luca.silvestri9@studio.unibo.it}}
	\end{itemize}
\end{document}